\documentclass{llncs}
\usepackage{fullpage}

%load needed packages
\usepackage{graphicx}
\usepackage{array}
\usepackage{booktabs}
\usepackage[utf8]{inputenc}


\begin{document}

\title{APPLICATION OF CLUSTERING METHODS TO
	SPORULATION YEAST MICROARRAY DATA}

\author{Diego De Pablo}
\institute{\email{depablodiego@uma.es} \\
Health Engineering. Málaga University.}

\maketitle 

\vspace{1cm} % Space down the title

\textit{Italic paragraph. Lorem ipsum dolor sit amet, consectetur adipiscing elit, sed do eiusmod tempor incididunt ut labore et dolore magna aliqua. Ut enim ad minim veniam, quis nostrud exercitation ullamco laboris nisi ut aliquip ex ea commodo consequat. Duis aute irure dolor in reprehenderit in voluptate velit esse cillum dolore eu fugiat nulla pariatur. Excepteur sint occaecat cupidatat non proident, sunt in culpa qui officia deserunt mollit anim id est laborum.}

% This is a comment


\section{Introduction}

\section*{Introduction}

The aim of this study is to apply clustering techniques to a DNA microarray dataset of \textit{Saccharomyces cerevisiae} gene expression during sporulation, and compare the results with those from a separate analysis team.

\subsection*{Reproduction and Life Cycle of \textit{S. cerevisiae}}

\textit{Saccharomyces cerevisiae} is a unicellular eukaryote commonly used as a model organism due to its biological processes resembling those of higher eukaryotes. It reproduces asexually through budding under favorable conditions, but under stress, it undergoes sporulation, involving meiosis and the formation of haploid spores. This adaptive strategy allows survival during environmental challenges, resuming growth when conditions improve.

\subsection*{Temporal Patterns in Gene Expression During Sporulation}

Gene expression during sporulation follows distinct temporal patterns, reflecting specific cellular events (reference 1). These include:
\begin{itemize}
	\item \textbf{Metabolic Early:} Rapid induction at t0.
	\item \textbf{Early I and II:} Sustained expression from t0.5 to t2.
	\item \textbf{Early-Middle:} Peak expression around t5.
	\item \textbf{Middle:} Activation between t5 and t7, related to meiosis.
	\item \textbf{Mid-Late:} Increased expression from t7 to t9, linked to spore wall formation.
	\item \textbf{Late:} Induction between t9 and t11.5, associated with spore maturation.
\end{itemize}

\subsection*{Application of Clustering Techniques to Analyze Sporulation Data}

Clustering algorithms, such as hierarchical clustering, K-means, and Diana, will be employed to classify genes based on their expression profiles across time points. This approach seeks to group genes into the known temporal classes and reveal insights into the transcriptional regulation during yeast sporulation.




\section{description of the methods}

\textbf{Bold paragraph. Lorem ipsum dolor sit amet, consectetur adipiscing elit, sed do eiusmod tempor incididunt ut labore et dolore magna aliqua. Ut enim ad minim veniam, quis nostrud exercitation ullamco laboris nisi ut aliquip ex ea commodo consequat. Duis aute irure dolor in reprehenderit in voluptate velit esse cillum dolore eu fugiat nulla pariatur. Excepteur sint occaecat cupidatat non proident, sunt in culpa qui officia deserunt mollit anim id est laborum.}

\section{most relevant results obtained comparing both methods}

\textbf{Bold paragraph. Lorem ipsum dolor sit amet, consectetur adipiscing elit, sed do eiusmod tempor incididunt ut labore et dolore magna aliqua. Ut enim ad minim veniam, quis nostrud exercitation ullamco laboris nisi ut aliquip ex ea commodo consequat. Duis aute irure dolor in reprehenderit in voluptate velit esse cillum dolore eu fugiat nulla pariatur. Excepteur sint occaecat cupidatat non proident, sunt in culpa qui officia deserunt mollit anim id est laborum.}
 
 
 \section{Conclusions}
 
 \textbf{Bold paragraph. Lorem ipsum dolor sit amet, consectetur adipiscing elit, sed do eiusmod tempor incididunt ut labore et dolore magna aliqua. Ut enim ad minim veniam, quis nostrud exercitation ullamco laboris nisi ut aliquip ex ea commodo consequat. Duis aute irure dolor in reprehenderit in voluptate velit esse cillum dolore eu fugiat nulla pariatur. Excepteur sint occaecat cupidatat non proident, sunt in culpa qui officia deserunt mollit anim id est laborum.}


\end{document}
